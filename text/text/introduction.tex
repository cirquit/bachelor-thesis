\chapter{Einführung}

\begin{itemize}
    \item Finden von Lösungen ohne große Anpassung von Hyperparamteren
    \item Domänen mit starken Einschränkungen (sparse Fitness, hochdimensionale kontinuierlicher Zustandsraum)
    \item Maschinelles Lernen ist eine potenziell große Hilfe zur Entwicklung von besseren Systemen
\end{itemize}

Die Nützlichkeit von Automatisierung ist offensichtlich
Viele benutzen Neuronale Netze mit Backpropagation, andere Möglichkeiten wie man Netze trainiert fallen ausm Fokus
Ich spreche ein paar Möglichkeiten an und untersuche sie auf ihre Nützlichkeit in Fälle wo Backprop nicht möglich ist


\section{Aufgabenstellung}
\begin{itemize}
    \item Verschiedene Techniken auf multi agenten systeme anzuwenden
    \item Kooperation und homogenität von Gewichten im Netz untersuchen
    \item Schauen ob ichs hinkriege
\end{itemize}

In dieser Arbeit beschäftige ich mich mit der Implementierung von verschiedenen Machine Learning Algorithmen für eine Domäne Half Field Offense mit sparsen Fitnesssignalen und einem hochdimensionalen kontinuierlichen Zustandsraum und stelle diese gegenüber. Aus dem Vergleich kann man die Nützlichkeit für andere Domänen mit ähnlichen Einschränkungen erahnen. 

\section{Motivation}
\begin{itemize}
    \item Praktikum - DARTS
    \item Gespräche mit Kommilitonen
    \item Abstrakte Zusammenhänge nutzen um Abhängigkeiten auf der Domänenseite zu abstrahieren
\end{itemize}

Im Siemens Praktikum bin ich das erste Mal mit Genetischen Algorithmen in Kontakt gekommen
Es kam die Frage auf welche Domänen ein Schwierigkeiten für Machine Learning Tasks darstellen und in der Industrie anwendbar sind
Verteilte Systeme, Roboter die zusammen etwas lernen, etc
Fitness Engineering, also die Anpassung von dem Rewardsignal, ist oft nötig für Aufgaben wo der Reward von viele Zeitschritten und kontinuierlichen Aktionsketten abhängt
Deshalb kam die Idee dies auf die Probe zu stellen, mit einem Fußballsimulator und neuen Algorithmen, die einen Durchbruch erreicht haben (CoSyNE)

\section{Aufbau der Arbeit}
\begin{itemize}
    \item Erklärung der Grundlagen, GA, NN, DCT, CoSyNE, CE
    \item Aufbaut der Domäne, Implementierung der Algorithmen
    \item Präsentation der Resultate
    \item Ausblick
    \item Die Implementierung und Problematiken sind im Appendix
    \item Veränderungen während der Arbeit
\end{itemize}


Im Rahmen dieser Arbeit werden im Kapitel 2 die Grundlagen von Genetischen Algorithen und deren Verknüpfung zu neuronalen Netzen und der Cross Entropy Method erklärt und anschaulich dargestellt. Kapitel 3 beschäftigt sich mit der Domäne, Parametrisierung der Algorithmen und der jeweiligen Resultate. Das Kapitel 4 gibt einen Ausblick in weitere Verbesserungsmöglichkeiten und legt verwandte Felder dar. Im Appendix wird die Implementierung von dem gesamten System überschlagen, Problematiken angesprochen und Lehren die für eine zukünftige Arbeit gezogen werden können angesprochen.