\chapter{Umsetzung in RoboCup2D}
    ``Was ist RoboCup'' \\
    ``Was für Ligen gibt es'' \\
    ``Was für eine Domäne ist es im Vergleich zu anderen''
    \section{Half Field Offense}
        ``Was ist Half Field Offense'' \\
        ``Wie ist das Spielfeld aufgebaut''
        \subsection{Zustandsraum}
            ``Unterschiedliche Zustandsräume, kommt drauf an wie viele Spieler aufm Feld sind (zitat vom HFO Paper)''\\
            ``Angepasst für Maschinen (zitat HFO)'' \\
            ``Umrechnung für Menschen''
        \subsection{Aktionsraum}
            ``Gibt 5 nicht parametrisierte Aktionen, die wir benutzt haben''\\
            ``Gibt noch andere parametrisierte''
        \subsection{Einschränkungen}
            ``Sparse Fitness''                 \\
            ``Simulation Learning''            \\
            ``Hochdimensional Kontinuierlich'' \\
            ``Auch genannt Black Box RL''

    \section{Implementierung der Algorithmen}
        ``Server in Haskell, HFOServer in C++, Agenten in Python''
        \subsection{Wahrscheinlichkeitsverteilung von Aktionen}
            ``Einfache Kodierung von Aktionen durch eine Wahrscheinlichkeitsverteilung, der Agent sampelt raus''\\
            ``Parametrisierung''
        \subsection{Cross Entropy mit DCT}
            ``Parametrisierung''
        \subsection{Neuroevolution mit DCT}
            ``Parametrisierung''
        \subsection{CoSyNE mit DCT}
            ``Parametrisierung''

    \section{Resultate}
        ``Die Resultate wurden an einem PC mit Specs ausgeführt''
        \subsection{1v1}
            ``Domäne nochmal erklären, Zustandsraum spezifizieren, sodass alles replizierbar ist''
            \subsubsection*{Wahrscheinlichkeitsverteilung}
                ``Graph zeigen, Interpretation bzw. Erklärung''
                ``Nach ~ 25 Episoden stagniert''
            \subsubsection*{Cross Entropy}
                ``Graph zeigen, Interpretation bzw. Erklärung''
                ``Aggressivität''
                ``Nach ~ 50 Episoden stagniert''
            \subsubsection*{Neuroevolution}
                ``Graph zeigen, Interpretation bzw. Erklärung''
            \subsubsection*{CoSyNE}
                ``Graph zeigen, Interpretation bzw. Erklärung''
        \subsection{Vergleich}
            ``Sicherheit <-> Aggressivität steht im Kontra zur Stabilität der Algorithmen''\\
            ``Wenn Zeit ein Faktor wäre'' \\
            ``Wenn Sicherheit ein Faktor wäre'' \\


