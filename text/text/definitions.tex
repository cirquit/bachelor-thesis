\chapter{Definitionen}

% \begin{itemize}
%     \item Einführung in ML Algorithmen
%     \item ``Im folgenden werden wir ein paar Möglichkeiten in Betracht ziehen wie man Genetische Algorithmen für Black Box Optimierung benutzen kann''
%    \item ``Ausserdem verknüpfen wir diese mit einer Reduzierung vom Suchraum durch Fouriertransformationen und verwenden verschiedene Kodierung der Individuen, durch Normalverteilungen und direkt durch Zahlen''
% \end{itemize}

Dieses Kapitel bietet Einblick in die Grundlagen von \textbf{Genetischen Algorithmen} im Zusammenhang mit \textbf{neuronalen Netzen} und der \textbf{Cross Entropy Method}. Außerdem werden einige Verbesserungen zu den naiven Methoden besprochen, wie die Reduzierung des Suchraums durch \textbf{Fouriertransformationen} und die Einführung von einer \textbf{kooperativen Evolution} durch Hinzufügen von einer neuen Methode zum GA.

    \section{Genetische Algorithmen}
        ``Die Motivation von Genetischen Algorithmen kam aus der Natur bla blubb''\\
        ``Im Folgenden behandeln wir die grundlegenden Operationen die einen GA ausmachen''
        \subsection{Selektion}
            ``Die besten x prozent werden genommen''
        \subsection{Kreuzung}
            ``Es gibt viele Kreuzungsoperationen, wir behandeln nur welche am Beispiel von einer Liste'' \\
            ``one-point, two-point und n-point crossover''
        \subsection{Mutation}
            ``Mutation ist hilfreich um die Varianz der Population etwas zu erhöhen'' \\
        \subsection{Repopulation}
            ``Ist ein Schritt der oft implizit beim Crossover passiert, (zitat finden), wird nicht nur benutzt um Kinder und Eltern zu verknüpfen, sondern auch völlig neue Individuen hinzuzufügen''

    \section{Neuroevolution}
        ``Neuroevolution beschäftigt sich mit der Verknüpfung von Genetischen Algorithmen und Neuronalen Netzn''
        \subsection{Neuronale Netze}
                ``Neuronale Netze sind eine riesige Gleichung''\\
                ``Formel von Machine Learning zeigen'' \\
                ``Backpropagation wird angesprochen aber nicht ausführlich erklärt''\\
            \subsubsection*{Dense Ebene}
                ``Kleines Beispielnetz aufmalen und durchrechnen''
            \subsubsection*{LSTM Ebene}
                ``Kleines Beispielnetz aufmalen und durchrechnen''
            \subsubsection*{Softmax Ebene}
                ``Erklärung bieten wieso es sowas gibt und was für eine Formel angewendet wird''
        \subsection{Verbindung mit genetischen Algorithmen}
            ``Die Verknüpfung findet in der Kodierung von den Individuen statt - wir nehmen eine (naive) Darstellung von allen Gewichten''
            ``Problematik -> Folgerung zu DCT''

    \section{Diskrete Cosinus Transformation - DCT}
        ``Der Suchraum kann durch die sog. DCT auf beliebige Dimensionalität eingeschränkt werden, wenn bestimmte Annahmen getroffen werden können'' \\
        ``Fouriertransformationen machen folgendes...'' \\
        ``Es gibt eine Inverse die aus einem n-stelligen liste eine m-stellige macht, wo die Zahlen korelliert sind (Beispiele)''
        \subsection{Kodierung des Suchraums}
            ``Die Kodierung besteht aus eine 20-stellingen Liste wie im Paper (Cosyne), wo damit 2k Gewichte entwickelt wurden''

    \section{Cooperative Synapsen Neuroevolution - CoSyNE}
        ``CoSyNE wurde vom Prof.Dr.Schmidhuber an der ETH Zürich entwickelt und hat damit sehr viele anderen Algorithmen in den Schatten gestellt''\\
        ``Methodik, ist wie GA bloß mit einer Aktion mehr die statt spielbare `Policies', nur im Koeffizientraum entwickle'' \\
        ``Dies erlaubt eine (zitat) Kooperative Entwicklung von Koeffizienten für die nachfolgenden Inv.DCT'' \\
        ``Kein Crossover, geringe Mutation'' \\
        ``Beispiele vom Erfolg'' \\
        \subsection{Permutation}
            ``Wir transponieren, shuffeln und transponieren zurück''

    \section{Cross Entropy}
        ``Cross Entropy ist eine andere Möglichkeit um die Individuen darzustellen, anstatt von Zahlen, hab ich nun pro Coeffizient eine Normalverteilung habe mit Mean und STD'' \\
        \cite{boer05}

        \subsection{Normalverteilung}
            ``Was ist eine Normalverteilung'' \\
            ``Wie programmiere ich eine Normalverteilung selber, Box-Muller, Randomness, Haskellcode Beispiel'' \\
