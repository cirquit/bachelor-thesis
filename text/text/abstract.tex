% -------------------------------------------------------------------------------------------------
%      MDSG Latex Framework
%      ============================================================================================
%      File:                  abstract.tex
%      Author(s):             Michael Duerr
%      Version:               1
%      Creation Date:         30. Mai 2010
%      Creation Date:         30. Mai 2010
%
%      Notes:                 - Place your abstract here
% -------------------------------------------------------------------------------------------------
%
\vspace*{2cm}

\begin{center}
    \textbf{Abstract}
\end{center}

\vspace*{1cm}

\noindent Wir untersuchen in dieser Bachelorarbeit verschiedene Ansätze zur Entwicklung von \textbf{neuronalen Netzen} am Beispiel der \textbf{Cross Entropy Method}, \textbf{genetische Algorithmen} und \textbf{CoSyNE} unter Einschränkung von spärlichen Fitnesssignalen, hochdimensionalen kontinuierlichen Zustandsräumen und simulationsbasierter Optimierung. \\[2mm]
\noindent
Der Suchraum wird durch \textbf{diskrete Kosinustransformationen} unter der Annahme reduziert, dass benachbarte Gewichte in neuronalen Netzen zueinander korreliert sind. Die Domäne ist eine Fußballsimulation, \textbf{Half Field Offense}, die Teams aus dem weltweiten Wettbewerb RoboCup2D mitliefert, an denen wir uns messen können. \\[2mm]
\noindent
Dafür entwickeln wir mehrere Angreifertaktiken im \textbf{1 gegen 1 Szenario} gegen dem Torwart aus der Standardimplementierung. Die Umsetzung erfolgt in Haskell und Python. \\[10mm]


\noindent
We analyze different approaches for \textbf{Neuroevolution}, by means of the \textbf{Cross Entropy Method}, \textbf{Genetic Algorithms} and \textbf{CoSyNE} with the restriction of sparse fitness signals, continous state spaces and simulation-based optimization. \\[2mm]
\noindent
The state space will be reduced with the help of \textbf{Discrete cosine transformation} under the assumption of correlated weights in neural nets. The domain is a soccer simulator, \textbf{Half Field Offense}, which includes teams from the worldwide competition RoboCup2D. \\[2mm]
\noindent
We develop different offensive policies for the \textbf{1 versus 1 scenario} with the goal keeper from the standard implementation. The languages used were Haskell and Python.






