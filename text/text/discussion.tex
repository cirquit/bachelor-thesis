\addtocontents{toc}{\protect\newpage}
\chapter{Diskussion}
    In diesem Kapitel studieren wir uns die Ergebnisse aus dem Vergleich zwischen den verschiedenen Algorithmen und deren Einschränkungen, erwägen den potenziellen Nutzen für ähnliche Probleme und stellen Verbesserungsvorschläge dar. Das Schlusswort umfasst verwandte Felder, die im Bezug zu unseren Algorithmen stehen.
    \section{Anwendungsmöglichkeiten}
        Unsere Aufgabe hat Einschränkungen, die für viele Probleme aus der realen Welt zutreffen, wie seltene Fitnesssignale, kontinuierliche und verrauschte Zustandsräume, sowie aufeinander aufbauende Aktionsketten. Daher glauben wir dass die Anwendungsgebiete, wie autonomes Verhalten voneinander abhängigen Robotern, selbstfahrende Autos oder das Entdecken von neuen Lösungswegen mithilfe von simplen Fitnessfunktionen, mit diesen Techniken weitergebracht werden kann. \\

        \noindent
        Vorallem war es interessant zu sehen, dass ANNs entwickelt werden können, die lediglich aus 20 Koeffizienten bestehen, extrem korellierte Gewichte produzieren und trotzdem erfolgreich sind. Ohne die Tests könnte man aus Abbildung \ref{fig:dct-my-case} ableiten, dass es extrem unwahrscheinlich ist, dass das befüllte Netz in irgendeiner Art Nutzen bringt. Deshalb können wir die Annahme, dass Gewichte in neuronalen Netzen zueinander korelliert sind, bestätigen \cite{cosyne1}. Es wäre vergleichsweise interessant zu schauen, ob bei Netzen die viel größer sind, ähnliche Ergebnisse möglich sind. \\

        \noindent
        CoSyNE wirft mit der treppenförmigen Fitnesskurve die Idee auf, ob eine stetige Verbesserung, oder explizit keine Verschlechterung als Garantie gegeben werden kann. Diese Eigenschaft wurde bisher nicht in der Literatur \cite{cosyne1}\cite{cosyne2}\cite{cosyne3} erwähnt und wäre interessant unter dem statistischen Sicherheitsaspekt zu untersuchen.

\newpage

    \section{Ausblick} \label{ausblick}
        In diesem Abschnitt behandeln wir konkrete Verbesserungsvorschläge der Algorithmen aus Kapitel \ref{basics} und interessante Thematiken für zukünftige Arbeiten, die aus zeitlichen Gründen nicht in diese Bachelorarbeit eingehen konnten.

        \subsection*{Genetische Algorithmen}
            Die genetische Suche wurde mit üblichen Parametern durchgeführt, die keine besondere Spezialisierung für die Anwendung bekommen haben. Darunter fällt der Mutationsparameter in CoSyNE, der in der Literatur sehr hoch gewählt war \cite{cosyne2} und die Begrenzung der Koeffizienten auf [-3,3] die wir gewählt so haben, weil es keine dafür Quellen dafür gab. % Ob die Größe dieser Zahlen eine Bedeutung hat, oder lediglich die Unterschiede innerhalb.

        \subsection*{Aufbau und Kodierung des neuronalen Netzes}
            Der Aufbau des ANNs kann bezüglich der Schicht aus LSTM-Neuronen in Tiefe und Größe verändert werden, da wir damit potenziell bessere Lernfähigkeiten erzielen können. Vorallem in der Kombination mit der DCT-Kompression besteht die Möglichkeit Netze mit über 1 Million Gewichten erfolgreich zu kodieren \cite{cosyne4}. \\

            \noindent
            Außer DCT gibt es noch eine andere Technik zur Verringerung des Suchraums, die auf Wavelets basiert und vielversprechende Ergebnisse in vielen Arcade Learning Environments geschafft habt \cite{wavelet}. Die Wavelet-Transformation funktioniert ähnlich wie die Fourier-Transformation und bildet den Suchraum (von Gewichten) auf einen Koeffizientenraum ab. Dabei wird aber nicht nur die Frequenz kodiert, wie bei DCT, sondern auch die Position der Daten, sodass wir eine stetige Ordnung beibehalten. In der Anwendung zur Kodierung von ANNs kann man aus der Lokalitätsinformation ableiten, zu welchem Neuron ein bestimmtes Gewicht gehört. Damit können wir uns auf bestimmte Bereich im Netz konzentrieren, anstatt das gesamte Netz mit dem veränderten Frequenzspektrum zu beeinflussen.\\

            \noindent
            Ein Vergleich zwischen DCT und der Wavelet-Transformation sieht man an der Octopus-Arm-Aufgabe \cite{wavelet}. Dabei muss man einen Arm, der aus 18 Segmenten besteht, in einer 2D Umgebung einen Punkt berühren lassen, indem die entsprechenden ``Muskeln'' angespannt werden. Dieses Problem wird oft als Benchmark für Neuroevoltion benutzt \cite{cosyne1}\cite{cosyne4}\cite{wavelet}.

%            Die Wavelets kodieren den Zustandsraum, genauso wie DCT, in hochfrequenten Domänen, behalten aber im Gegensatz zu Fouriertransformationen die Stetigkeit in der Ordnung der Daten, was zu schnelleren Lösungen in der Octopusarm Aufgabe geführt hat.

        \subsection*{Cross Entropy Method}
            Die Cross Entropy Method Lösung wurde lediglich in der einfachsten Form umgesetzt und es findet weder Repopulation mit neuen Individuen statt, noch haben wir einen Mutationsschritt. Das Hinzufügen und Spezialisierung von diesen Methoden würde wahrscheinlich zu besseren Lösungen führen. Viele Verbesserungen finden sich dazu in der Literatur bei der Spezialisierung für das Spiel Tetris \cite{cem}.

        \subsection*{Aktionsraum}
            Der High-Level Aktionsraum wurde während der Arbeit im HFO Framework erweitert, sodass wir Aktionen wie \textit{Go\_To\_Ball} oder \textit{Reduce\_Angle\_To\_Ball} nicht benutzen konnten. Da die Dokumentation der übrigen Aktionen sehr spärlich ausgefallen ist, würden eigene Aktionen eine bessere Möglichkeit bieten über die Effektivität der Algorithmen zu argumentieren. Sie würden auch sehr wahrscheinlich vielartige Taktiken zulassen. \\

            \noindent
            Die Low-Level Aktionen mit Argumenten, wie \textit{Pass} oder \textit{Kick\_To}, könnten in dem ANN mit geringem Aufwand als zusätzliche Ausgangsparameter modelliert werden und würden eine schwierigere, aber dennoch interessante Herausforderung darstellen.


        \subsection*{Multi-Agenten Systeme}
            Da die Simulation skalierend modelliert wurde, erlaubt sie uns einfaches Testen mit mehreren Agenten, wie 2vs1, oder 4vs4. Leider hätten ausführliche Tests dieser Domänen den zeitlichen Rahmen der Bachelorarbeit maßlos gesprengt. 

\newpage

    \section{Schlusswort}
        Das Schlusswort bilden wir aus dem Anriss über Forschungsthemen, dem HFO Framework und den fortlaufenden Nutzen dieser Arbeit für die aktuelle Forschung.

        \subsection*{Implementierung für OpenAI Gym}
            Während der Entwicklung dieser Arbeit gab es eine Implementierung der HFO Domäne für das 1vs1 Szenario in der OpenAI Gym, das ein bekanntes Machine Learning Framework in Python ist. Die Umsetzung von unseren Algorithmen dafür wäre der nächste logische Schritt um die neuroevolutionären Methoden mehr in den Fokus zu Rücken. Diese spezielle Implementierung erlaubt uns Schritte vorzusimulieren, da der Torwart deterministisch ist. Damit können andere Trainingstechniken, wie Backpropagation, benutzt werden, die einen interessanten Vergleich anstellen.\\

            \noindent
            Eine andere interessante Fragestellung ist ob das Wissen, das nach dem Lernen gegen den vordefinierten Torwart entstanden ist, auch gegen den Weltmeister aus dem Jahr 2012 ``Helios'' anwendbar ist.


        \subsection*{ConvNet, CoSyNE, DCT und Wavelets}
            In dem Paper von Jan Schmidhuber \cite{cosyne4} wird CoSyNE und DCT für die Komprimierung von über 1 Million Gewichten in einem convolutional neural network benutzt und hat beeindruckende Ergebnisse für die Steuerung von einem Auto mit Bilddaten erzielt. Deshalb würde es sich vielleicht anbieten die Verknüpfung von CoSyNE und DCT zum Vortrainieren von Modellen für convolutional networks auszuprobieren. \\

            \noindent
            Wavelets, die von Jürgen Schmidhuber als logischer Nachfolger für die DCT Transformation im Gewichtssuchraum entwickelt wurden \cite{wavelet}, sollten auch einem Vergleich mit naiven GAs und CoSyNE unterzogen werden, da bei unseren Beispielen der naive GA bessere Individuen entwickelt hat.

        \subsection*{RoboCup}
            RoboCup hat sich als Ziel gesetzt echte Roboter gegen die Fußballweltmeister im Jahr 2050 antreten zu lassen und ich bin nach dieser Arbeit der Ansicht, dass es gar nicht so unwahrscheinlich ist. Wenn wir die aktuellen Robotikfortschritte anschauen, wie Atlas von Boston Dynamics \cite{robot}, der bereits externe Krafteinwirkungen abfangen kann, ist der Weg nicht mehr weit zu gemeinsamen sportlichen Aktivitäten.\\

            \noindent
            Man muss sich vor Augen führen, dass die Entwicklung von Machine Learning in den letzten 12 Jahren einen extremen Sprung gemacht hat, was Bilderkennung \cite{NIPS2012_4824}, eine Go KI \cite{go}, Stimmsynthese \cite{wavenet} und allgemeine Spieler für das Atari Arcade Learning Environment \cite{Naddaf2010}, angeht. Wir sind heute noch ganze 34 Jahre von der Deadline entfernt und wenn wir mit ähnlicher Effektivität diese Zeit nutzen werden wie die letzten Jahre, bin ich davon überzeugt, dass Fußball eine eigene Roboter Liga bekommen wird.
