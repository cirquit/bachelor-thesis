\addtocontents{toc}{\protect\newpage}
\chapter{Diskussion}
    In diesem Kapitel schauen wir uns die Ergebnisse aus dem Vergleich zwischen den verschiedenen Algorithmen und deren Einschränkungen an, erwägen den potenziellen Nutzen für ähnliche Probleme und stellen Verbesserungsvorschläge dar. Das Schlusswort umfasst verwandte Felder die im Bezug zu unseren Algorithmen stehen.
    \section{Anwendungsmöglichkeiten}
        Unsere Aufgabe hat Einschränkungen, die für viele Probleme aus der realen Welt zutreffen, wie seltene Fitnesssignale, kontinuierliche und verrauschte Zustandsräume, sowie aufeinander aufbauende Aktionsketten. Daher glauben wir dass die Anwendungsgebiete, wie autonomes Verhalten voneinander abhängigen Robotern, selbstfahrende Autos oder das Entdecken von neuen Lösungswegen mithilfe von simplen Fitnessfunktionen, mit diesen Techniken weitergebracht werden kann. \\

        \noindent
        Vorallem war es interessant zu sehen, dass KNNs entwickelt werden können, die lediglich aus 20 Koeffizienten bestehen, extrem korellierte Gewichte produzieren und trotzdem lernen können. Ohne die Tests könnten man aus Abbildung \ref{fig:dct-my-case} ableiten, dass es extrem unwahrscheinlich ist, dass das befüllte Netz in irgendeiner Art Nutzen bringt. Deshalb können wir die Annahme, dass Gewichte in neuronalen Netzen keine große Abweichung zueinander haben müssen, bestätigen \cite{cosyne1}. Es wäre interessant zu schauen, ob bei Netzen die viel größer sind, ähnliche Ergebnisse möglich wären. \\

        \noindent
        CoSyNE wirft mit der treppenförmigen Fitnesskurve die Idee auf, ob eine stetige Verbesserung, oder explizit keine Verschlechterung als Garantie gegeben werden kann. Diese Eigenschaft wurde bisher nicht in der Literatur erwähnt und wäre unter dem statistischen Sicherheitsaspekt interessant zu untersuchen.

\newpage

    \section{Ausblick} \label{ausblick}
        Diese Arbeit hat leider einen festen Zeitrahmen gehabt und wir konnten vieles nicht ausprobieren. Wir sprechen mögliche Verbesserungen an, die man in zukünftigen Arbeiten beachtet werden können.

        \subsection*{Genetische Algorithmen}
            Die genetische Suche wurde mit ausgewogenen Parametern durchgeführt, die jedoch keine besondere Spezialisierung für die Anwendung bekommen haben. Darunter fällt der Mutationsparamter in CoSyNE, der in der Literatur sehr hoch gewählt war \cite{cosyne2} und die Begrenzung der Koeffizienten auf [-3,3] die wir gewählt haben, weil es keine Literatur dafür gab. 
        \subsection*{Aufbau des neuronalen Netzes}
            Der Aufbau des KNNs kann mit von der Schicht aus LSTM Neuronen in Tiefe und Größe verändert werden, da wir potenziell bessere Lernfähigkeiten bekommen können. Vorallem in der Kombination mit der DCT Kompression besteht die Möglichkeit Netze mit über 1 Million Gewichten erfolgreich zu kodieren \cite{cosyne4}. \\[2mm]
            \noindent
            Ausser DCT gibt es noch eine andere Suchraumverringerung die auf Wavelets basiert und vielversprechendere Ergebinsse in vielen Arcade Learning Environments geschafft hat \cite{wavelet}. Die Wavelets kodieren den Zustandsraum auch in hochfrequenten Domänen, behalten aber im Gegensatz zu Fouriertransformationen die Stetigkeit in der Ordnung der Daten, was zu schnelleren Lösungen in der Octopusarm Aufgabe geführt hat.

        \subsection*{Cross Entropy Method}
            Die Cross Entropy Method Lösung wurde lediglich in der einfachsten Form umgesetzt und es fand weder Repopulation mit neuen Individuen statt, noch haben wir einen Mutationsschritt gehabt. Das Hinzufügen von diesen Methoden würde wahrscheinlich zu besseren Lösungen führen. Viele Verberbesserungen finden sich auch in \cite{cem}.

        \subsection*{Aktionsraum}
            Der High-Level Aktionsraum wurde während der Arbeit im HFO Framework erweitert, sodass wir Aktionen wie \textit{Go\_To\_Ball} oder \textit{Reduce\_Angle\_To\_Ball} nicht benutzt haben. Da die Dokumentation der übrigen Aktionen sehr spärlich ausgefallen ist, würden eigene Aktionen eine bessere Möglichkeit bieten über die Effektivität der Algorithmen zu argumentieren. Sie würden auch sehr wahrscheinlich mehr Taktiken zulassen.

        \subsection*{Multi-Agenten Systeme}
            Da die Simulation skalierend modelliert wurde, erlaubt sie uns einfaches Testen mit mehreren Agenten, wie 2vs1, oder 4vs4. Leider hätten ausführliche Tests dieser Domänen den zeitlichen Rahmen der Bachelorarbeit maßlos gesprengt.

\newpage

    \section{Schlusswort}
        TODO


        \subsection*{Implementierung für OpenAI Gym}
            Während der Entwicklung dieser Arbeit gab es eine Implementierung der HFO Domäne für das 1vs1 Szenario in der OpenAI Gym, das ein bekanntes machine learning Framework in Python ist. Die Umsetzung von unseren Algorithmen dafür wäre der nächste logische Schritt um die neuroevolutionären Methoden mehr in den Fokus zu Rücken. Diese spezielle Implementierung erlaubt uns Schritte vorzusimulieren, da der Torwart deterministisch ist. Damit können andere Trainingstechniken, wie Backpropagation, benutzt werden, die einen interessanten Vergleich anstellen.

        \subsection*{ConvNet, CoSyNE, DCT und Wavelets}
            In \cite{cosyne4} wird CoSyNE und DCT für die Komprimierung von über 1 Million Gewichten in einem Convolutional Neural Network benutzt und hat beeindruckende Ergebnisse für die Steuerung von einem Auto mit Bilddaten erzielt. Deshalb würde es sich anbieten die Verknüpfung von CoSyNE und DCT auf anderen Aufgaben, die auf hochdimensinalen korrelierten Daten basieren, anzuwenden. \\

            \noindent
            Wavelets, die von Jürgen Schmidhuber als als logischer Nachfolger für die DCT Transformation im Gewichtssuchraum entwickelt wurden \cite{wavelet}, sollten auch einem Vergleich mit naiven GAs und CoSyNE unterzogen werden, da bei unseren Beispielen der naive GA bessere Individuen entwickelt hat.

        \subsection*{RoboCup}
            RoboCup hat sich als Ziel gesetzt echte Roboter gegen die Fußballweltmeister im Jahr 2050 antreten zu lassen und ich bin nach dieser Arbeit der Ansicht, dass es gar nicht so unwahrscheinlich ist. Wenn wir die aktuellen Robotikfortschritte anschauen, wie Atlas von Boston Dynamics \cite{robot}, der bereits externe Krafteinwirkungen abfangen kann, ist der Weg nicht mehr weit zu gemeinsamen sportlichen Aktivitäten.\\

            \noindent
            Man muss sich vor Augen führen, dass die Entwicklung von Machine Learning in den letzten 12 Jahren einen extremen Sprung gemacht hat, was Bilderkennung \cite{NIPS2012_4824}, eine Go KI, Stimmsynthese und allgemeine Spieler für das Atari Arcade Learning Environment \cite{Naddaf2010}, angeht. Wir sind heute noch ganze 34 Jahre von der Deadline entfernt und wenn wir mit ähnlicher Effektivität diese Zeit nutzen werden wie die letzten Jahre, bin ich überzeugt davon, dass Fußball eine eigene Roboter Liga bekommen wird.
